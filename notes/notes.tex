\documentclass[12pt, letterpaper]{article}
\usepackage{graphicx}
\usepackage[a4paper, total={6in, 10in}]{geometry}
\usepackage{amsmath}
\usepackage{bm}
\usepackage{listings}
\usepackage{xcolor}
\lstset { %
    language=C++,
    backgroundcolor=\color{black!5}, % set backgroundcolor
    basicstyle=\footnotesize,% basic font setting
}

\title{Algorithm and Data Structure Analysis Notes}
\author{Sam Kirk} 
\date{July 2024}

\begin{document}
\maketitle

LOOK AT CLUBS FROM FIRST LECTURE  

\section*{Week 1}
Include time complexity notation cheat sheet 

\section*{Week 3}
\subsection*{Radix Sort}
Put the last digit of each element into buckets and sort it. 
Repeat for all the digits until fully sorted. Can do first digit 
as well (MSD). Order is O(dn+dk) for n numbers on d digits 
that range from 1 to k. 

\subsection*{Counting sort}
O(n+k) = O(n) for n inputs in the range of 1 to k.  

\subsection*{Order Statistics}
The ith order statistic of n elements is the ith smallest element. 
The minimum is the 1st, and the maximum element is the nth order 
statistic. The median is the n/2 order statistic. If n is even there 
are two medians. 

\subsection*{Randomized Selection}
This is a method to find the ith element statistic. 
So imagine we want to find the 5th smallest element in an array of 
size 11. We randomly select a pivot, say 9, and since 9>5, we 
put the two largest elements in the array at index 10 and 11. 
Now we call the function again, with the size 9, still looking for 
index 5. If we select a new pivot of 2, we put the two smallet 
elements at the front and call the function again with size 9-2 =7. 
Here we change the index we are looking for to 3, since we chopped 
off the first two smallest elements. We keep going until we land 
on the 5th smallest element. 

\section*{Week 4}

\section*{Week 5}
\subsection*{AVL Trees}

\section*{Week 6}
\subsection*{Hasing With Chaining}
Also known as Open hashing or Closed adressing.
Hashing with chaining is a collision resolution technique where 
each table slot contains a linked list of elements that hash to the 
same index. When a collision occurs (multiple keys mapping to the same 
hash value), the new element is appended/preneded to the list.

\subsection*{Hashing with Linear Probing}
Here when a collision occurs, the algorithm searches for the next 
available slot by moving linearly through the table. 

\subsection*{Chaining vs Probing}
Chanining maintains referential integrity. It potentially takes 
up a lot of space. Linear probing is a contiguous block of memory 
which aligns well with modern processors. It does get slower 
as the table fills up though.  



\section*{To include in cheat sheet for exam}
\begin{itemize}
    \item Graph algorithms and how they work 
\end{itemize}

\end{document}